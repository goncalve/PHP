\documentclass{article}
% Pour les caract�res accentues et les titres de sections
\usepackage[latin1]{inputenc}
\usepackage[francais]{babel}
\usepackage[T1]{fontenc}
\usepackage{listings}
\usepackage{color}
\usepackage{textcomp}
\definecolor{fondcode}{rgb}{0.95,0.95,0.95}
\definecolor{violet}{rgb}{0.5,0,0.5}
\lstset{upquote=true}


%------------------------------
%---- la taille de la page ----
%------------------------------
\voffset 0.5cm
\hoffset 0.7cm
\textheight 22cm
\textwidth 14.6cm
\topmargin 0.0cm
\oddsidemargin 0.0cm
\evensidemargin 0.0cm

% Pour les en-t�tes et pieds de pages
\usepackage{fancyhdr}
\pagestyle{fancy}
\usepackage{lastpage}
\renewcommand\headrulewidth{1pt}
\fancyhead[L]{IUT A. Lyon1\\
Informatique}
\fancyhead[C]{Langage PHP\\
2�me ann�e}
\fancyhead[R]{TD 01 - PHP, Transmission de donn�es\\
Correction}
\renewcommand\footrulewidth{1pt}
\fancyfoot[C]{\textbf{- \thepage/\pageref{LastPage} -}}
\fancyfoot[R]{\today}

\title{TD 01 -PHP, Transmission de donn�es} 
\author{Isabelle Gon�alves} 

\begin{document}
%
\large\bf{Exercice 1 : helloV1.php}

\lstset{language=HTML, commentstyle=\color{blue}\textit, backgroundcolor=\color{fondcode}, keywordstyle=\color{red}\bfseries, basicstyle=\ttfamily\small, showstringspaces=false, stringstyle=\color{cyan}, breaklines=true, tabsize=1}

\begin{lstlisting}
<!DOCTYPE html> 
<HTML>
<HEAD>
   <meta charset="utf-8">
   <TITLE>Des Hellos</TITLE> 
</HEAD>

<BODY>
  <ul>
    <?php
     for($i=0; $i<10; $i++)
     {
         echo '  <li> Hello World! </li>'; 
         echo "\n";  /* pour introduire des sauts de ligne 
                          dans le code HTML g�n�r�        */
     }
    ?>
  </ul> 
</BODY> 
</HTML>

\end{lstlisting}

\small Ne pas oublier que le code g�n�r� doit pour chaque exercice passer le valideur HTML5: \\
 http\string://validator.w3.org/ \\

\large\bf{Exercice 2 : helloV2.php}

\lstset{language=HTML, commentstyle=\color{blue}\textit, backgroundcolor=\color{fondcode}, keywordstyle=\color{red}\bfseries, basicstyle=\ttfamily\small, showstringspaces=false, stringstyle=\color{cyan}, breaklines=true, tabsize=1}

\begin{lstlisting}
<!DOCTYPE html>
<HTML>
<HEAD>
	<TITLE>Des Hellos 2</TITLE>
</HEAD>
<ul>
<BODY>
  <?php
   if (isset($_GET['nbhello'])) /* teste que le param�tre 
                                 est pass� dans l'URL */
   {
     $nb = (int) $_GET['nbhello']; // neutralisation par cast
     if ($nb>0 && $nb<=100)
       for($i=0; $i<$nb; $i++)
         echo '<li> Hello World! </li>';
     else 
         echo '<h1> Valeur nbhello incorrecte dans l\'URL !</h1>';
   } 
   else
     echo '<h1> Pas de valeur de nbhello dans l\'URL !</h1>';
  ?>
</ul>
</BODY>
</HTML>

\end{lstlisting}

\begin{small}
  
A tester sans param�tre dans l'URL: \\
http...helloV2.php\\
A tester avec un param�tre dans la gamme attendue dans l'URL: \\
http...helloV2.php?nbhello=15\\
A tester avec un param�tre hors de la gamme attendue dans l'URL: \\
http...helloV2.php?nbhello=150\\
\end{small}

\large\bf{Exercice 3 : helloV3.php}

\begin{lstlisting}
<!DOCTYPE html>
<HTML>
<HEAD>
	<TITLE>Des Hellos 3</TITLE>
</HEAD>
<BODY>

  <FORM action="helloV3.php" method="post">
    <label for="nbhello">Combien de Hello ?</label>
    <input type="text" id="nbhello" name="nbhello" size="3" maxlength="3" autofocus 
    <?php 
      if(isset($_POST['nbhello'])){
        $nb=(int)$_POST['nbhello'];
        echo ' value="'.$nb.'"';}
     // cette ligne ci-dessus ajoute le champ value 
     // si le formulaire a �t� valid�
     // afin de garder la valeur tap�e.
    ?>
    /> <!-- Fin de la balise input -->
  </FORM>

  <ul>
  <?php
  if (isset($nb))
  {
    if ($nb>0 && $nb<=100)
      for($i=1; $i<=$nb; $i++)
      {
        echo '<li> Hello World! </li>';
        if($i%10==0)
          echo '<li>'.$i.'</li>';
      }
    else 
      echo '<h1> Valeur incorrecte !</h1>';
  }
  ?>
  </ul>
</BODY>
</HTML>

\end{lstlisting}

\begin{small}
  
Il ne faut pas d'affichage d'erreur quand la page est charg�e la premi�re fois.
Il faut tester en ne proposant pas de valeur, ou une valeur en dehors de la gamme attendue.\\
Remarque : le fait de taper sur la touche "entr�e" �quivaut � valider le formulaire. On aurait aussi pu mettre un bouton "valider" dans le formulaire.\\

\end{small}

\large\bf{Exercice 4 : helloV4.php}

\begin{lstlisting}
<!DOCTYPE html>
<HTML>
<HEAD>
	<TITLE>Des Hellos 4</TITLE>
	<link rel="stylesheet" href="helloV4.css"/>
</HEAD>
<BODY>
<FORM action="helloV4.php" method="post">
  <label for="nbhello">Combien de Hello ?</label>
  <input type="text" id="nbhello" name="nbhello" size="3" maxlength="3" autofocus 
    <?php if(isset($_POST['nbhello'])){
      $nb=(int)$_POST['nbhello'];
      echo ' value="'.$nb.'"';}
     // cette ligne ci-dessus ajoute le champ value si le formulaire 
     // a �t� valid� afin de garder la valeur tap�e.
    ?>
  /> <!-- Fin de la balise input -->
</FORM>

  <ul>
<?php
    if (isset($nb))
    {
    	if ($nb>0 && $nb<=100)
    	  for($i=1; $i<=$nb; $i++)
    	  {
		    if($i%2==0)
		      echo '<li class="paire"> Hello World! </li>';
		    else
		      echo '<li class="impaire"> Hello World! </li>';
		    if($i%10==0)
		      echo '<li>'.$i.'</li>';
		  }
		else 
		  echo '<h1> Valeur incorrecte !</h1>';
	}
?>

  </ul>
</BODY>
</HTML>

\end{lstlisting}

\large\bf{Exercice 4 : helloV4.css}

\begin{lstlisting}
.paire
{
  color: blue;
}

.impaire
{
  color: red;
}
\end{lstlisting}

\large\bf{Exercice 5 : impots.php}

\lstset{language=HTML ,commentstyle=\color{blue}\textit, backgroundcolor=\color{fondcode}, keywordstyle=\color{red}\bfseries, basicstyle=\ttfamily\small, showstringspaces=false, stringstyle=\color{cyan}, breaklines=true, tabsize=1}

\begin{lstlisting}
<!DOCTYPE html>
<HTML>
<HEAD>
	<TITLE>Pr�vision des imp�ts</TITLE>
</HEAD>
\end{lstlisting}
\lstset{
  language        = php,
  basicstyle      = \small\ttfamily,
  keywordstyle    = \color{blue},
  stringstyle     = \color{violet},
  identifierstyle = \color{magenta},
  commentstyle    = \color{cyan},
  emph            =[1]{php},
  emphstyle       =[1]\color{black},
  emph            =[2]{if,and,or,else},
  emphstyle       =[2]\color{red}}
\begin{lstlisting}
<?php
  // Partie calcul � s�parer quand c'est possible de l'affichage
    if(isset($_POST['nbenfants']) AND isset($_POST['salaire']))
    {   
    	$nb=(int)$_POST['nbenfants']; //neutralisation par cast
        $sal=(float)$_POST['salaire']; //neutralisation par cast
        // calcul du quotient familial
    	$parts = $nb/2+1;
    	if (isset($_POST['mariage']))
    	  $parts++;
    	$qf = (0.72 * $sal)/$parts;
        // calcul de l'imp�t
    	$imp=0;
        $bornesMin=array(66680, 24873, 11199, 5615);
        $taux=array(0.4, 0.3, 0.14, 0.055);
        $nbTranchesImposees=4;

        for($i=0;$i<$nbTranchesImposees; $i++)
        {
          if ($qf>=$bornesMin[$i])
          {
            $tranche=$qf -($bornesMin[$i]-1);
            $imp+=$tranche*$taux[$i];
            $qf-=$tranche;
          }
        }
    	$imp=$imp*$parts; 
    }
?>
\end{lstlisting}

\lstset{language=HTML ,commentstyle=\color{blue}\textit, backgroundcolor=\color{fondcode}, keywordstyle=\color{red}\bfseries, basicstyle=\ttfamily\small, showstringspaces=false, stringstyle=\color{cyan}, breaklines=true, tabsize=1}

\begin{lstlisting}
<!-- Partie affichage -->
<BODY>
<FORM action="impots.php" method="post">
  <p><label for="nbenfants">Nombre d'enfants :</label>
  <input type="text" name="nbenfants" id="nbenfants" size="3" maxlength="3" autofocus required 
  <?php if(isset($nb))
    echo ' value="'.$nb.'"'; ?>
  /> <!-- Fin de la balise input -->
  </p>
  <p><label for="mariage">Mari� :</label>
  <input type="checkbox" name="mariage" id="mariage" 
  <?php if(isset($_POST['mariage']))
    echo ' checked'; ?>
  /> <!-- Fin de la balise input -->
  </p>
  <p><label for="salaire">Salaire annuel :</label>
  <input type="text" name="salaire" id="salaire"  size="30" maxlength="30" autofocus required 
  <?php if(isset($sal))
    echo ' value='.$sal; ?>
  /> <!-- Fin de la balise input -->
  </p>
  <input type="submit" value="calculer" />
</FORM>

<?php
  if(isset($imp))
	  echo '<p>Votre imp�t pr�visionnel est de : '. $imp . ' &euro;';
?>

</BODY>
</HTML>


\end{lstlisting}




\end{document}